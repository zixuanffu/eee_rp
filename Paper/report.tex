\documentclass[12pt]{article}
\usepackage{fullpage,graphicx,psfrag,amsmath,amsfonts,verbatim}
\usepackage[small,bf]{caption}
\usepackage{amsthm}
% \usepackage[hidelinks]{hyperref}
\usepackage{hyperref}
\usepackage{bbm} % for the indicator function to look good
\usepackage{color}
\usepackage{mathtools}
\usepackage{fancyhdr} % for the header
\usepackage{booktabs} % for regression table display (toprule, midrule, bottomrule)
\usepackage{adjustbox} % for regression table display
\usepackage{threeparttable} % to use table notes
\usepackage{natbib} % for bibliography
\usepackage{tikz}
\usetikzlibrary{arrows.meta}
\input newcommand.tex
\bibliographystyle{apalike}
% \setlength{\parindent}{0pt} % remove the automatic indentation

\title{When to book a flight?\\{\large {A response to dynamic pricing and uncertainty}}}
\author{Fu Zixuan \\{\small {Research proposal}}}
\date{\today}

\begin{document}
\maketitle
% \thispagestyle{empty}
% \begin{abstract}

% \end{abstract}

% \newpage
% \thispagestyle{empty}
% \tableofcontents
% \newpage

% \setcounter{page}{1}

\section{Introduction}
It is a well-known fact that airline companies adopt dynamic pricing to
optimize revenue. The same strategy applies to other markets with perishable
inventories as well, such as tickets for entertainment events. It has been
estimated that dynamic pricing increases airline's revenue by 7.6\% compared to
uniform pricing \citep{williams2022welfare}. In addition, many economic models
have studied the optimal pricing strategy for firms, assuming that consumer
arrivals follow a Poisson process with time-varying willingness to pay and
price elasticity.

Though it is not without good reason to model the consumer side in this way and
ignore forward-looking behaviors, it is in my own interest to shed light on the
traveler's intertemporal decision—what and when to buy? Anecdotal experience
suggests that travelers may start checking prices a few months ahead but wait
until a certain time to purchase. While prices may be low early on, they delay
purchase until they are more certain about their travel plans. While the entire
price change is unknown to travelers at the time of decision, they form some
expectations about the price dynamics to make this kind of trade-off between
cost and uncertainty. In fact, by setting a Google flight tracker, one can
observe the ups and downs of the prices of desired flights, which fluctuate
more or less around a low level at first and then increase as the departure
date approaches.

In this paper, I aim to model the consumer's dynamic decision-making process as
each individual solving an optimal stopping problem. By doing so, I seek to
answer the question of how consumers trade off between minimizing costs and
reducing uncertainty. Through estimating the parameters that govern their
decision-making, I plan to conduct counterfactual analyses on consumer welfare
where firms adjust their pricing strategies (e.g., removing uncertainty and
implementing only a monotonically increasing price structure) or modify their
flexibility options.

The paper is structured as follows: Section 2 reviews the related literature
and highlights the unique features of the problem addressed in this paper. In
Section 3, I present the model, starting with a simplified case. Section 4
discusses the additional complexities that can be incorporated into the model.

\section{Related Literature}
This section revisits several classical dynamic discrete choice problems. By
classifying and summarizing these examples, I aim to highlight the properties
of the problem addressed in this paper and the ways in which it differs from
existing literature.

\paragraph{Base Case: One-Time Purchase of Non-Storable Goods}
In the base case, a consumer enters the market, makes a purchase decision \(a\)
based on product characteristics \(x\) and personal attributes \(z\), and then
exits the market. The purchased goods are intended for immediate consumption
and cannot be stored. Consequently, there is no inventory, and the
decision-making process is static, with no intertemporal considerations. For
example, studies that make this assumption include those on cereal purchases
\citep{nevo2001measuring}, or car purchases \citep{berry1995automobile}.

\paragraph{Repeated Decision with Consumption Utility for Storable Goods}
This scenario introduces \textbf{stockpiling} behavior, as seen in the purchase
of goods like detergent \citep{hendel2006measuring} or soft drinks. Here, the
consumer enters the market with an existing inventory level \(i_t\) and faces
individual inventory costs or limits. Their purchase decision depends on both
the product characteristics \(x_t\) and the current inventory \(i_t\). For
instance, the product price fluctuates between a regular price \(p_r\) and a
sale price \(p_s\). At time \(t\), if the consumer chooses to buy, they pay the
current price and accumulate inventory to consume over time, incurring holding
costs or facing storage constraints. If they choose not to buy, they consume
from their existing inventory.

The dynamic aspect arises from the trade-off between purchasing at a lower
price (e.g., during sales) and incurring the cost of holding additional
inventory. The consumer must weigh the immediate cost savings against the
future costs and benefits associated with stockpiling.

\paragraph{One-Shot Decision with Flow Utility for Durable Goods}
In this case, the consumer makes a one-time purchase decision, after which they
accrue holding utility from the durable good over time. For example, consider
purchasing a car \citep{grigolon2018consumer} or installing solar panels
\citep{de2019subsidies}.\footnote{Patent renewal \citep{pakes1984patents} is
    similar, though distinct, as the decision-maker exits by not paying, whereas in
    the purchase example, the agent exits by buying.} The decision involves a
trade-off between current costs and future benefits. The consumer evaluates
whether the present price justifies the expected long-term utility of owning
the product, considering potential changes in future costs or benefits.

\paragraph{Repeated Decision with Flow Utility for Durable Goods}
Here, consumers face a recurring decision to purchase or replace a durable
good, such as in the bus engine replacement problem \citep{rust1987optimal} or
the camcorder industry \citep{gowrisankaran2012dynamics}. At any given time,
they hold and use only one product. If they decide to purchase, the holding
utility derives from the newly acquired product \(f_j\); otherwise, it comes
from the existing product \(f_0\).

In this scenario, the characteristics of the owned product evolve over time
(e.g., depreciation, as in the bus engine case), while market options may
improve (e.g., technological innovation, as in the camcorder industry). Each
period, the consumer evaluates whether to purchase a new product now (incurring
an upfront cost but benefiting from higher quality) or delay the purchase
(accepting lower holding utility in the short term but potentially gaining from
lower future prices or better-quality options). This dynamic decision-making
process involves balancing current utility against expected future gains.

\paragraph{One-Shot Decision with Final Consumption Utility for Perishable Goods}
This case primarily applies to goods like event tickets
\citep{sweeting2012dynamic} or transport tickets
\citep{board2016revenue,gershkov2018revenue}. Such goods have no residual value
after their designated usage period, making them "perishable." Since ticket
sales typically begin well in advance of the final usage date, consumers often
start considering purchases several periods before the deadline.

The consumer purchases only one ticket and consumes it at the final usage date,
meaning there is no ongoing utility from holding the ticket. The decision is
not about balancing current versus future flow utility but rather about
minimizing costs under price fluctuations. Consumers anticipate price trends
(e.g., initial variability followed by an eventual increase) and may delay
their purchase to secure a lower price.

Additionally, the time gap between purchase and consumption introduces
uncertainty. As the final usage date approaches, the consumer gains better
information about their plans or needs, allowing them to make a more informed
decision. Thus, the trade-off involves evaluating price dynamics alongside the
uncertainty of future utility at the time of purchase.

The next section will formalize the dynamic problem the consumer faces when
balancing cost and uncertainty in each period after they enter the market
(begin checking prices). First, I will consider a simplified case and then
gradually introduce more complexities.

\section{Model}

I focus on airline ticket purchase in this section. To illustrate the idea,
figure \ref{fig:toy_model} depicts the simplest case. In this toy model with
discrete time, I denote the last period as \(t=0\), which corresponds to the
departure date. The value of \(t\) represents the number of periods before
departure.

% \begin{figure}
%     \centering
%     \includegraphics[width=0.8\textwidth]{toy_model.png}
%     \caption{Timeline of airline ticket purchase}
%     \label{fig:toy_model}
% \end{figure}

\begin{figure}[h!]
    \centering
    % Resize the figure to 80% of the text width
    \begin{tikzpicture}[>=Stealth]
        % Axes
        \draw[->] (-1,0) -- (8,0) node[below] {}; % x-axis
        \draw[->] (-1,0) -- (-1,6) node[left] {Price}; % y-axis

        % Timeline labels
        \node[below] at (0,0) {Dec};
        \node[below] at (3,0) {Jan};
        \node[below] at (6,0) {Feb};

        % Points and arrows
        \node at (0,2) {$V_2 = \max\{x\beta+\epsilon_2, \E(V_1)\}$};
        \draw[->] (0,1.5) -- (0,1);

        \node at (3,3) {$V_1$};
        \draw[->] (3,2.5) -- (3,2);

        \node at (6,5) {$V_0 = \max\{x\beta + \epsilon_0, 0\}$};
        \draw[->] (6,4.5) -- (6,4);

        % Connecting arrows
        \draw[->] (0,1) -- (3,2);
        \draw[->] (3,2) -- (6,4);
    \end{tikzpicture}

    \caption{Timeline of airline ticket purchase}
    \label{fig:toy_model}
\end{figure}

\paragraph{Background} Consider a traveler who wants to fly in February 2025 (\(t=0\)). Suppose that
in December 2024 (\(t=2\)), the traveler enters the market by searching for
flights online. She then sets a price alert on Google Flights tracker
\footnote{Add citation or URL for Google Flights tracker here.} for all the
flights around that time. In the following two months (\(t = 2, 1,0\)), she
decides whether to buy based on the current price \(p_t\) and her travel plans.
She has a binary choice $a$, fly or not to fly. Therefore, her choice set is
defined by \(\{0,1\}\). If she purchases a ticket, she will only realize her
consumption utility at the last period (\(t=0\)), which corresponds to
\(x\beta\).

For example, it might be the case that in December, she buys the ticket
immediately at price \(p_{t2}\). As time progresses to January $t=1$, the
situation evolves as she realizes that she would be occupied at that time. If
she were to make a decision now $t=1$, she would not fly. However, the ticket
she bought at $t=2$ is non-modifiable and non-refundable (unfortunately). When
she reaches \(t=0\), she realized utility $x\beta+\epsilon_{t0}$.

\paragraph{State} At time \( t \), the known state of the individual includes:
\begin{itemize}
    \item \( p_t \): The current price of the ticket.
    \item \( \epsilon_t \): The random shock observed in period \( t \).
    \item \( \text{Bought}\): Whether the ticket has already been purchased.
    \item \( x \): The product characteristics that determine the utility of flying.
\end{itemize}
The uncertainty arises from the next period \( t-1 \) price and shock.
\begin{itemize}
    \item The future price $p_{t-1}$. I assume that the price goes up in, following AR(1)
          process with $p_{t-1}=\rho p_t + \nu_{t-1}$ and $\rho>1$.
    \item The future shock $\epsilon_{t-1}$. I assume that they are T1EV with variance
          \(\sigma_t^2\) and $\sigma_{t}^2>\sigma_{t-1}^2$. The variance goes down as the
          departure date approaches.
\end{itemize}
\paragraph{Action}
\begin{itemize}
    \item \( a_t = 1 \): Buy the ticket in period \( t \).
    \item \( a_t = 0 \): Wait until the next period to decide.
\end{itemize}

\paragraph{State Transition}\
\begin{itemize}
    \item If \( a_t = 1 \): The decision process ends, and no further actions are
          possible.
    \item If \( a_t = 0 \): The state transitions to the next period \( t - 1 \), and a
          new shock \(\epsilon_{t-1}\) is observed.
\end{itemize}

\paragraph{Rewards}
\begin{itemize}
    \item If \( a_t = 1 \): Immediate reward is:
          \[
              R_t(a_t = 1) = V_0 - \alpha p_t - \epsilon_1,
          \]
          where \( V_0 \) is the final utility of flying, adjusted by the ticket price
          and the shock at \( t \).
    \item If \( a_t = 0 \): No reward is received in the current period.
\end{itemize}

\paragraph{Value Function} The value function \( V_t \) represents the maximum expected utility at time \(
t \), given the state:
\[
    V_t = \max_{a_t} \mathbb{E}[R_t(a_t) + V_{t-1}].
\]
\begin{itemize}
    \item If \( a_t = 1 \): The process terminates with reward \( R_t(a_t = 1) \).
    \item If \( a_t = 0 \): The individual moves to \( t - 1 \), and the expectation
          incorporates uncertainty about the future.
\end{itemize}

\paragraph{Recursive Formulation}
The recursive relationship for the value function is:
\[
    V_t = \max \left\{ R_t(a_t = 1), \mathbb{E}[V_{t-1} | \text{no purchase}] \right\}.
\]

\begin{itemize}
    \item Terminal period $t=0$:
          \begin{itemize}
              \item If the ticket is not purchased, utility is zero. If purchased:
                    \[
                        V_0 = x\beta - \alpha p_0 - \epsilon_0.
                    \]
              \item Expected value at \( t = 0 \):
                    \[
                        \E[V_1] = \int\int \max\left\{x\beta - \alpha p_0 - \epsilon_0, 0\right\} f(\epsilon_0)f(p_0) d\epsilon_0dp_0,
                    \]
                    where \( f(\epsilon_0), f(p_0)\) is the probability density function of the
                    shock and the price.
          \end{itemize}
    \item Intermediate period $t>0$:
          \begin{itemize}
              \item  At \( t = 1 \), the value function is:
                    \[
                        V_1 = \max \left\{ x\beta - \alpha p_1 - \epsilon_1, \mathbb{E}[V_0] \right\}.
                    \]
              \item At \( t = 2 \), the value function is:
                    \[
                        V_2 = \max \left\{ x\beta - \alpha p_2 - \epsilon_2, \mathbb{E}[V_1] \right\}.
                    \]

          \end{itemize}
\end{itemize}

\paragraph{Choice Probability}
Each individual \( i \)'s decision is based on the dynamic program, where they
maximize their expected utility given the current information. The probability
of individual \( i \) taking action \( a_t = 1 \) is given by:
\[
    P(a_t = 1 | p_t) = \frac{\exp((x\beta - \alpha p_t) / \sigma_t)}{\exp((x\beta - \alpha p_t) / \sigma_t) + \exp(\mathbb{E}[V_{t-1}] / \sigma_t)}.
\]

\paragraph{Estimation}
We observe in the data the flight characteristics (in this case, only a flight
dummy), the price of the ticket, and the purchase decision at each period. The
parameters to be estimated include:
\begin{itemize}
    \item $\beta$: the coefficient on the flight characteristics.
    \item $\alpha$: the price coefficient.
    \item $\rho,\sigma_{\nu}$: the price process parameters.
    \item $\sigma_t$: the standard deviation of the shock at each period.
\end{itemize}

The most straightforward approach is to estimate by maximum likelihood with
individual purchase data. Other estimation techniques are to be investigated.

\section{Discussion}

The previous section presents the simplified scenario where the traveler makes
only a binary choice. To construct the full choice set for an individual who
has purchased a ticket, I include all the flights available with the same
origin and destination, with departure dates in proximity.

The market/choice set considered consists of flights by different companies,
with various characteristics (the number of layovers, duration, the departure
(arrival) time of the day, airline brand, etc.). With respect to layovers, I
would like to include both direct and connecting flights because many
international flights involve layovers. Since prices for international flights
are usually high and price changes can be up to 20\% \footnote{This is
    according to personal experience. The statistics need to be verified.}, it
gives travelers higher incentives to think about the timing of purchase. In
addition, for the same flight, there are usually different flexibility options,
which can be included in the $x_j$. Meanwhile, the flexibility property can
affect how the shock impacts consumers. Therefore, the size of the shock can be
modeled as a function of the flexibility of the ticket with
$\epsilon_t(a_t)=\epsilon_t(x_j(a_t))$, where $x_j(a_t)$ is the flexibility of
the ticket choice $a_t$.

In the toy model, (I assume that) the consumer assumes that the price follows
an AR(1) process. The price dynamics can be made more realistic, for example,
by assuming the price fluctuates around a certain level before a certain period
($t>t_c$) and then increases more or less monotonically.

In papers that focus on the firm side pricing strategy
\citep{williams2022welfare,betancourt2022dynamic}, it is a common assumption
that willingness to pay (WTP) consumers arrive early (leisure traveler) and
high WTP consumers arrive late (business traveler).\footnote{This pattern may
    be reversed in the case of event or performance ticket sales. Early buyers are
    often avid fans who closely follow the performers and assign greater value to
    the event, leading to a higher willingness to pay. In contrast, late buyers may
    discover the event later and have a lower willingness to pay.} This variation
in WTP over time can be incorporated into consumer utility models by
introducing a heterogeneous price coefficient that depends on both the time of
arrival and individual characteristics.

% In the data, we observe individual search and purchase behavior. The moment
% they enter the market is marked by the first price search. 
\pagebreak \newpage \bibliography{../References/ref.bib}

\end{document}